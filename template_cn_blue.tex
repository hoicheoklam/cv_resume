%# -*- coding:utf-8 -*-
%% start of file `template_en.tex'.
%% Copyright 2006-1008 Xavier Danaux (xdanaux@gmail.com).
%
% This work may be distributed and/or modified under the
% conditions of the LaTeX Project Public License version 1.3c,
% available at http://www.latex-project.org/lppl/.


\documentclass[11pt,a4paper]{moderncv}

\usepackage{fontspec,xunicode}
\usepackage[slantfont,boldfont]{xeCJK}
\usepackage{xcolor}  % replace by the encoding you are using
\definecolor{myblack}{RGB}{34,34,34}
\definecolor{mycolor}{RGB}{244,243,242}
% \setmainfont{Tahoma}
\setmainfont{Times New Roman}  % 缺省英文字体.serif是有衬线字体sans serif无衬线字体
\setCJKmainfont[ItalicFont={SimHei}, BoldFont={SimHei}, Path = ./, Extension = .ttf]{SimHei} 
\setCJKsansfont{STSong}
\setCJKmonofont{STFangsong}  % 中文等宽字体


%-----------------------xeCJK下设置中文字体------------------------------%
\setCJKfamilyfont{song}{SimSun}  % 宋体 song
\newcommand{\song}{\CJKfamily{song}}
\setCJKfamilyfont{fs}{FangSong_GB2312}  % 仿宋2312 fs
\newcommand{\fs}{\CJKfamily{fs}}
\setCJKfamilyfont{yh}{Microsoft YaHei}  % 微软雅黑 yh
\newcommand{\yh}{\CJKfamily{yh}}
\setCJKfamilyfont{hei}{SimHei}  % 黑体  hei
\newcommand{\hei}{\CJKfamily{hei}}
\setCJKfamilyfont{hwxh}{STXihei}  % 华文细黑  hwxh
\newcommand{\hwxh}{\CJKfamily{hwxh}}
\setCJKfamilyfont{asong}{Adobe Song Std}  % Adobe 宋体  asong
\newcommand{\asong}{\CJKfamily{asong}}
\setCJKfamilyfont{ahei}{Adobe Heiti Std}  % Adobe 黑体  ahei
\newcommand{\ahei}{\CJKfamily{ahei}}
\setCJKfamilyfont{akai}{Adobe Kaiti Std}  % Adobe 楷体  akai
\newcommand{\akai}{\CJKfamily{akai}}


%------------------------------设置字体大小------------------------%
\newcommand{\chuhao}{\fontsize{42pt}{\baselineskip}\selectfont}  % 初号
\newcommand{\xiaochuhao}{\fontsize{36pt}{\baselineskip}\selectfont}  % 小初号
\newcommand{\yihao}{\fontsize{28pt}{\baselineskip}\selectfont}  % 一号
\newcommand{\erhao}{\fontsize{21pt}{\baselineskip}\selectfont}  % 二号
\newcommand{\xiaoerhao}{\fontsize{18pt}{\baselineskip}\selectfont}  % 小二号
\newcommand{\sanhao}{\fontsize{15.75pt}{\baselineskip}\selectfont}  % 三号
\newcommand{\sihao}{\fontsize{14pt}{\baselineskip}\selectfont}  % 四号
\newcommand{\xiaosihao}{\fontsize{12pt}{\baselineskip}\selectfont}  % 小四号
\newcommand{\wuhao}{\fontsize{10.5pt}{\baselineskip}\selectfont}  % 五号
\newcommand{\subwuhao}{\fontsize{10pt}{\baselineskip}\selectfont}  % 次五号
\newcommand{\xiaowuhao}{\fontsize{9pt}{\baselineskip}\selectfont}  % 小五号
\newcommand{\liuhao}{\fontsize{7.875pt}{\baselineskip}\selectfont}  % 六号
\newcommand{\qihao}{\fontsize{5.25pt}{\baselineskip}\selectfont}  % 七号


% \usepackage{fontawesome}
% \setCJKmainfont[BoldFont={WenQuanYi Micro Hei/Bold}]{WenQuanYi Micro Hei}
% \defaultfontfeatures{Mapping=tex-text}
% \XeTeXlinebreaklocale "zh"
% \XeTeXlinebreakskip = 0pt plus 1pt minus 0.1pt
% moderncv themes
\moderncvtheme[burgundy]{classic}  % optional argument are 'blue' (default), 'orange', 'red', 'green', 'grey' and 'roman' (for roman fonts, instead of sans serif fonts)
% \moderncvtheme[green]{classic}  % idem
% \moderncvtheme[blue,roman]{hht}
% character encoding



% adjust the page margins
\usepackage[scale=0.9]{geometry}
% \setlength{\hintscolumnwidth}{3cm}  % if you want to change the width of the column with the dates
% \AtBeginDocument{\setlength{\maketitlenamewidth}{6cm}}  % only for the classic theme, if you want to change the width of your name placeholder (to leave more space for your address details
\AtBeginDocument{\recomputelengths}  % required when changes are made to page layout lengths

% personal data
\firstname{\textcolor{myblack}{\song 许}}
\familyname{\textcolor{myblack}{\song 卓岚}}
\title{CheokLam}  % optional, remove the line if not wanted
% \address{杭州}{}  % optional, remove the line if not wanted
\address{2005/12/09}{}  % optional, remove the line if not wanted
\mobile{+61 0423 932 568}  % optional, remove the line if not wanted
% \fax{fax (optional)}  % optional, remove the line if not wanted
\email{hoicheoklam@gmail.com}  % optional, remove the line if not wanted
%\homepage{Blog: http://geekplux.com}  % optional, remove the line if not wanted
\social[github]{https://github.com/hoicheoklam}
\extrainfo{%
  https://www.linkedin.com/in/cheok-lam-hoi/ \\
  WeChat: hcl051209 \\
}

\photo[72pt]{3.jpg}  % '64pt' is the height the picture must be resized to and 'picture' is the name of the picture file; optional, remove the line if not wanted
% \quote{Test}  % optional, remove the line if not wante

% \nopagenumbers{}  % uncomment to suppress automatic page numbering for CVs longer than one page


%----------------------------------------------------------------------------------
%            content
%----------------------------------------------------------------------------------
\begin{document}

\pagecolor{mycolor}

\maketitle
\vspace*{-10mm}

\section{\song {教育经历}}
\cventry{24.02-至今}{本科}{\song 墨尔本大学}{\song 计算机科学}{\song 澳大利亚墨尔本,}{}
\cvlistitem{\song 核心课程:数据库系统,面向对象编程,算法与数据结构,计算机系统基础,机器学习基础}

\vspace*{1mm}

\cventry{15.09-18.06}{高中}{\song 嘉诺撒圣心英文中学}{\song 高中文凭 - 理科方向}{\song 中国澳门}{}
\cvlistitem{\song 所获荣誉:学业优秀奖,杰出表现奖}

\section{\song {技能}}
\cvline{\textbf{编程语言}}{Python (Pandas, NumPy, scikit-learn, Matplotlib), Dart, Java, JavaScript, C/C++, SQL}
\cvline{\textbf{技术/框架}}{PyTorch, TensorFlow, Firebase, Flutter, Express.js, Node.js, React, Spring Boot}
\cvline{\textbf{开发工具}}{Git/GitHub, VS Code, Android Studio, Arduino IDE, Linux, MS Office, Blender}
\cvline{\textbf{语言}}{\song 粤语(母语), 普通话(母语), 英语(雅思8.0)}

\section{\song {工作经历}}
\cventry{24.12-25.02}{上海交通大学 Thinklab 实验室}{\song 科研实习}{}{}{}
\cvline{}{\song 参与基于深度学习的自动驾驶路径规划算法的研发,协助实现算法原型并评估其性能。}
\cvline{}{\song 应用K近邻算法构建多模态传感器数据分类器, 以提升自动驾驶系统在动态环境下的目标识别准确率。}
\cvline{}{\song 实现并对比评估了CNN、RNN及Transformer等主流深度学习模型在图像分类与描述生成任务上的性能。}
\cvline{}{\song 使用Python及其数据科学库进行数据预处理、分析与可视化, 提升了数据处理效率。}
\vspace*{1mm}

\cventry{24.04-25.02}{墨尔本大学 HackMelbourne}{\song 教育项目负责人}{https://hack.melbourne/}{}{}
\cvline{}{\song 组织并主持了3场线上技术讲座,涵盖前端开发、后端开发及全栈开发等主题,每场活动均吸引50+名参与者。}
\cvline{}{\song 运用React开发具备状态管理与响应式设计的UI/UX组件, 提升导航效率, 进而提高参与度。}
\cvline{}{\song 在黑客马拉松期间围绕代码调试、Git工作流与项目架构等环节进行辅导, 显著提升开发效率。}

\section{\song 项目经历}

\subsection{Fwend, \song 一款校园社交应用}
\cvline{Flutter}{\song 基于Flutter开发的跨平台移动应用,通过个性分析与实时地理位置匹配大学生用户。}
\cvline{Firebase}{\song 使用Firebase进行用户认证和数据存储,确保用户数据的安全性和隐私保护,实时数据同步与云存储功能,测试阶段支持100+用户并发访问。}
\cvline{Node.js}{\song 后端使用Node.js与Express.js构建RESTful API,实现用户数据管理与匹配逻辑,确保高效响应与可扩展性。}
\cvline{Git}{\song 使用Git进行版本控制,管理代码库并协同开发,提升团队协作效率。}
\vspace*{1mm}
\subsection{\song 无线健康监测与紧急报警系统}
\cvline{Arduino, C++ VirtualWire}{\song 基于Arduino (C++) 构建实时生理监测系统, 负责心率与压力数据的采集。采用模块化代码结构与VirtualWire库设计通信协议, 增强了系统可维护性与扩展性。}
\cvline{}{\song 通过优化底层数据传输协议,将无线传输延迟降低至毫秒级,显著提升了数据实时性。}
\cvline{}{\song 在接收端实现事件驱动架构, 成功将危急警报的触发响应时间缩短60\%, 达到200毫秒内, 保障了系统的及时性。}



% \subsection{Vocational}
% \cventry{year--year}{Job title}{Employer}{City}{}{Description}  % arguments 3 to 6 are optional
% \cventry{year--year}{Job title}{Employer}{City}{}{Description}  % arguments 3 to 6 are optional
% \subsection{Miscellaneous}
% \cventry{year--year}{Job title}{Employer}{City}{}{Description line 1\newline{}Description line 2}% arguments 3 to 6 are optional

% \section{Languages}
% \cvlanguage{language 1}{Skill level}{Comment}
% \cvlanguage{language 2}{Skill level}{Comment}
% \cvlanguage{language 3}{Skill level}{Comment}

% \section{Computer skills}
% \cvcomputer{category 1}{XXX, YYY, ZZZ}{category 4}{XXX, YYY, ZZZ}
% \cvcomputer{category 2}{XXX, YYY, ZZZ}{category 5}{XXX, YYY, ZZZ}
% \cvcomputer{category 3}{XXX, YYY, ZZZ}{category 6}{XXX, YYY, ZZZ}

% \section{Interests}
% \cvline{篮球}{\small 体力与技巧}
% \cvline{hobby 2}{\small Description}
% \cvline{hobby 3}{\small Description}

% \renewcommand{\listitemsymbol}{-}  % change the symbol for lists

% \section{Extra 1}
% \cvlistitem{Item 1}
% \cvlistitem{Item 2}
% \cvlistitem[+]{Item 3}  % optional other symbol% XeLaTeX can use any Mac OS X font. See the setromanfont command below.
% Input to XeLaTeX is full Unicode, so Unicode characters can be typed directly into the source.

% The next lines tell TeXShop to typeset with xelatex, and to open and save the source with Unicode encoding.

% !TEX TS-program = xelatex
% !TEX encoding = UTF-8 Unicode

% \section{Extra 2}
% \cvlistdoubleitem[\Neutral]{Item 1}{Item 4}
% \cvlistdoubleitem[\Neutral]{Item 2}{Item 5}
% \cvlistdoubleitem[\Neutral]{Item 3}{}

%% Publications from a BibTeX file
% \nocite{*}
% \bibliographystyle{plain}
% \bibliography{publications}  % 'publications' is the name of a BibTeX file

% \begin{thebibliography}{}
% \bibitem[]{} 移动增强现实可视化综述[C]. ChinaVis 2017.
% \end{thebibliography}


\end{document}


%% end of file `template_en.tex'.

%%% Local Variables:
%%% mode: latex
%%% TeX-command-extra-options: "-shell-escape"
%%% TeX-master: t
%%% TeX-engine: xetex
%%% End: